
\section{Installing PLOW}

Currently, there are two ways to obtain PLOW:

\bi
\item {\em Using a version of XSB configured with PITA}.  If you have,
  or can make a version of XSB that is configured with PITA, simply
  invoke PLOW like any other package via the command:

\begin{verbatim}
?- [plow].
\end{verbatim}

Note that the first time PLOW is used, XSB will automatically download
PLOW from

{\tt https://github.com/theresasturn/plow.git}.

\item {\em Using a Docker Image}.  This approach is recommended for
  users who are not proficient in building XSB, although it does
  require some facility in using Docker.  For this approach simply
  download an XSB Docker image from Docker Hub ({\tt
    https://hub.docker.com}), via the commands

\begin{verbatim}
docker login
docker pull theresasturn/xsb
\end{verbatim}

Information about how to use Docker is widely available on the web.
For those not familiar with Docker, a quick start is as follows.  Run
the image via the command:

\begin{verbatim}
docker run -it  theresasturn/xsb
\end{verbatim}

At this point your command line will be within the docker image (as
user {\tt root}).  You can now start XSB with the command:

\begin{verbatim}
/xsb-src/bin/xsb
\end{verbatim}

At this point, you can invoke PLOW like any other package via the
command:

\begin{verbatim}
?- [plow].
\end{verbatim}

Note that the first time PLOW is used, XSB will automatically download
PLOW from

{\tt https://github.com/theresasturn/plow.git}.

You can exit from XSB using the usual {\tt ?- halt} command, and exit
from the Docker session via the command {\tt exit}.  At this point any
work you have done (such as cloning PLOW) will have been saved into a
{\em container}, and the container should be used in the future
whenever you may be changing any files in it.  The container's name can be found using the command 

\begin{verbatim}
docker ps -a
\end{verbatim}

and then looking for the {\em NAMES} column corresponding to the {\em
  IMAGE} {\tt theresasturn/xsb} (e.g., keen\_wright).  Using that
name, the continer can be reinvoked using the command.

\begin{verbatim}
docker start -i keen_wright
\end{verbatim}

Files can be easily copied into the Docker container via the
command\footnote{Note that additional software can be installed into a
  container via the command {\tt apt-get install -y}, e.g., {\tt
    apt-get install -y emacs}.}

\begin{verbatim}
docker cp <filename> keen_wright:<path-name>
\end{verbatim}

\ei

