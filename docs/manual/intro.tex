
This  manual  describes  how  to  use  FPLP,  a  meta-interpreter  for
performing  quantitative and/or  multi-valued  reasoning using  tabled
logic programming.  In FPLP, a ground atom $A$ can be associated with
a {\em strength} $\sigma$ using  the syntax $A \sim \sigma$.  What the
strength  strength represents  depends on  the actual  semantics used.
For  instance,  the stength  can  be a  number  that  denotes, say,  a
probability or  a value in  fuzzy logic.  Alternatively,  the strength
may be  an atom denoting, say,  a qualitative likelihood  such as {\em
``possible''} or {\em ``highly likely''}.  Depending on the semantics,
the  propagation of  strengh  values through  logical expressions  and
rules varies.  The major groups of semantics that are supported are as
follows.

\begin{itemize}
\item {\em Lattice-based Semantics} In a semantics bassed on a
complete lattice $L$, each strength $s_i$ is associated with an
element of $L$.  The strength of a conjunction $A \sim s_1,B \sim s_2$
is the meet in $L$ of $s_1$ and $s_2$, while the strength of a
disjunction $A \sim s_1;B \sim s_2$ is the join in $L$ of $s_1$ and
$s_2$.  And finally, the default negation of a literal $naf{} A \sim
s_1$ is complement in $A$ of $s_1$.
%
\item {\em T-norm based Semantics}.  In a semantics based on a T-norm $T$
(and its associated -conorm) each strength $s_i$ is associated with an
element of the real number interval $[0,1]$.  The strength of a
conjunction $A\sim s_1,B\sim s_2$ is $T(s_1,s_2)$; the strength of a
disjunction $A\sim s_1;B\sim s_2$ is {\em co-}$T(s_1,s_2)$.  And
finally, the default negation of a literal $naf{} A\sim s_1$ is
 $1 - s_1$.

As will be discussed below, T-norm based semantics are well-suited for
reasoning with vague concepts (such as whether a given person is
tall); and with reasoning over knowledge that is based on similarity
or relevancy measures.  Reasoning about such statements often requires
the use of a quantitative strength that is not probabilistic in the
formal sense.  The current implementation of FPLP supports the min,
product, and Lukasiewicz T-norms.
%
\item {\em Probability-based Semantics}.  Last, but not
least \ourtool{} also supports probabilistic semantics for logic
programs based on the distribution semantics~\cite{}.  Both of these
semantics are supported using the syntax of Logic Programs with
Annotated Disjunctions (LPADs).  The probabilistic semantics supported
include: 
\bi
\item {\em The full distribution semantics~\cite{}}, which extends
logic programs with full Bayesian reasoning.  (In fact logic programs
under the full distribution semantics are strictly more powerful than
Bayesian nets.)  However, like probabilistic reasoning in general,
probabilistic logic programming can have a high computational cost.

\item {\em The restricted distribution semantics}.  For applications
that don't need the generality of the full distribution
semantics, \ourtool{} also offers an implementation of the restricted
distribution semantics, which makes the assumption of the independence
of occurrence of probabilsitic atoms within a derivation, along with
an assumption of the exclusivity of multiple derivations of a subgoal
with respect to probabilistic atoms encountered by each derivation.
\end{itemize}
\ei 

FPLP implements these extensions within a very general logic
programming framework for knowledge representation that includes:

\bi
%
\item Specification of fully recursive rules that include both default
negation {\tt naf} and explicit negation {\tt neg}.

\item The ability to specify the strength of quantitative rules in a
  variety of functions of the strength of their bodies, including 
  strength boosting, strength decay, and sigmoid functions.
%
\item Evaluation of these rules according to the three-valued
Paraconsistent Well-Founded Semantics (WFS)
\ei
