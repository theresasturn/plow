\subsubsection{Dynamic Stratification} \label{dyn-strat}

One of the most important formulations of stratification is that of
{\em dynamic} stratification.  \cite{Przy89d} shows that a program has
a 2-valued well-founded model iff it is dynamically stratified, so
that it is the weakest notion of stratification that is consistent
with the well-founded semantics.
%
%As presented in~\cite{Przy89d}, dynamic stratification computes strata
%via operators on \emph{3-valued interpretations} -- pairs of the form
%$\langle Tr;Fa \rangle$, where $Tr$ and $Fa$ are subsets of the Herbrand
%base $\cH_P$ of a normal program $P$.
As presented in~\cite{Przy89d}, dynamic stratification computes strata
via operators on interpretations of the form $\langle Tr;Fa \rangle$,
where $Tr$ and $Fa$ are subsets of $\cH_P$.
%
%-----------------------------------------------------------------------
\begin{definition} \label{def:dyn-ops}
For a normal program $P$, sets $Tr$ and $Fa$ of ground atoms and a
3-valued interpretation $I$ (sometimes called a pre-interpretation):
\begin{description} \item[$True^P_I(Tr) =$]
%                $\{A:val_I(A) \neq {\tt t}$
    $\{A|A$ is not true in $I$;  and 
                        there is a clause
                        $B \leftarrow L_1,...,L_n$
                in $P$, a grounding substitution $\theta$ such that
                $A = B\theta$ and for every $1 \leq i \leq n$ either
                $L_i\theta$ is true in $I$, or $L_i\theta \in Tr$\};
  \item[$False^P_I(Fa) =$] 
%$\{A : val_I(A) \neq {\tt f}$ 
$\{A|A$ is not false in $I$; and for every
    clause $B \leftarrow L_1,...,L_n$ in $P$ and grounding substitution
    $\theta$ such that $A = B\theta$ there is some $i$ $(1 \leq i \leq
    n)$ such that $L_i\theta$ is false in $I$ or $L_i\theta \in Fa\}$.
\end{description}
\end{definition}
%------------------------------------------------------------------------()     
%
\cite{Przy89d} shows that $True^P_I$ and $False^P_I$ are both
monotonic, and defines $\kcaltrue^P_I$ as the least fixed point of $True^P_I(\emptyset)$
and $\calfalse^P_I$ as the greatest fixed point of
$False^P_I(\cH_P)$.
%\footnote{Below, we will sometimes omit the program $P$ in
%  these operators when the context is clear.}.
%, along with an
%operator $\cal I$ that assigns to every interpretation $I$ of $P$ a
%new interpretation ${\cal I}(I) = I \cup \langle \cT_I ; \cF_I \rangle$.
%
In words, the operator $\kcaltrue^P_I$ extends the interpretation $I$ to add
the new atomic facts that can be derived from $P$ knowing $I$; $\calfalse^P_I$
adds the new negations of atomic facts that can be shown false in $P$
by knowing $I$ (via the uncovering of unfounded sets).  An iterated
fixed point operator builds up dynamic strata by constructing
successive partial interpretations as follows.
%----------------------------------------------------------------------          
\begin{definition}[Iterated Fixed Point and Dynamic Strata]
\label{def:IFP}
For a normal program $P$ let 

\begin{center}
$  \begin{array}{rcl}
          WFM_0 & = & \langle \emptyset ; \emptyset \rangle;      \\
 WFM_{\alpha+1} & = &       WFM_{\alpha} \cup
                                \langle \kcaltrue^P_{WFM_\alpha};\calfalse^P_{WFM_\alpha} \rangle; \\
     WFM_\alpha & = & \bigcup_{\beta < \alpha} WFM_\beta, \mbox{ for limit ordinal }\alpha.
  \end{array}
$
\end{center}

\noindent
  $WFM(P)$ denotes the fixed point interpretation $WFM_\delta$,
  where $\delta$ is the smallest (countable) ordinal such that both
  sets $\kcaltrue^P_{WFM_\delta}$ and $\calfalse^P_{WFM_\delta}$ are empty.
%($\delta$ exists, and is
%  a countable ordinal because both $\kcaltrue_I$ and $\calfalse^P_I$ are monotonically
%  increasing).  
% We refer to $\delta$ as the {\em depth} of program $P$.  
The {\em stratum} of atom $A$, is the least ordinal $\beta$ such that
   $A \in WFM_{\beta}$.
% (where $A$ may be either in the true or false
%   component of $WFM_{\beta}$).
\end{definition}
%------------------------------------------------------------------------()      
%
\cite{Przy89d} shows that %the iterated fixed point 
$WFM(P)$ is in fact the well-founded model and that any undefined
atoms of the well-founded model do not belong to any stratum --
i.e. they are not added to $WFM_{\delta}$ for any ordinal
$\delta$. Thus, a program is \emph{dynamically stratified} if every
atom belongs to a stratum.

%------------------------------------------------------------------------
% taking out fixed-order stuff until we put LPADs in (though maybe we
% wont even need it then )

%\input{fixed-order-dynstrat}
