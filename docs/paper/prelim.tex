\section{Semantics}\label{sec:semantics}
%
{\sc First Pass: assume fixed Godel T norm -- we can extend to other
  tnorms and to distribution semantics later.  Even after simplifying,
  there are several land mines to navigate.

TODO!!! 
Unfounded sets. and continue with the definition below.
}

\subsection{Preliminaries}
%We assume a general knowledge of logic programming terminology,
%including tabled resolution and the well-founded semantics.  In
%addition we make use of the following terminology and assumptions.

An {\em annotated atom} is an atom $A$ associated with an annotation
$n$ that is either a variable or $0 \leq n \leq 1$, denoted $A:n$.
From an annotated atom {\em A:n}, an {\em objective literal} $O$ is
formed as either $O = A:n$, termed a positive objective literal with
$sign(O) = pos$; or aa $O = \ourneg A:n$, termed a negative objective
literal with $sign(O) = neg$.  In either case the annotation, $n$, is
denoted as $ annotation(O)$, while the underlying atom, $A$, is
$atom(A)$.
%
Two objective literals $O_1$ and $O_2$ with the same underlying atom
are {\em homologs} if they have the same sign and {\em conjugates}
otherwise.  An objective literal $A:n$ is ground if both $A$ and $n$
are ground.
%
% = A:n$ and $O_2 = \neg A:m$ are termed {\em conjugates}.
%denoted as $O_1 = conjugate(O_2)$ or $O_2 = conjugate(O_1)$; 
%Otherwise if $atom(O_1) = atom(O_2)$ $O_1$ and $O_2$ are termed {\em
%  homologs}.
%
%, denoted as $O_1 = homolog(O_2)$ or $O_2 = homolog(O_1)$.
From an objective literal $O$ a default literal is formed as either a
positive default literal $O$ or as a negative default literal $\naf
O$.  A default literal is sometimes simply called a {\em literal}.
\footnote{
When convenient, an objective literal $A:1$ ($\neg A:1$) is denoted
simply as $A$ ($\neg A$).}

A rule has the form
\[r = O \mif{} L_1,\ldots,L_n\]
where $O$ is an objective literal and $L_0,\ldots,L_n$ are default
literals.

A rule $R$ is ground if all literals in $R$ are ground; a program
$\cP$ is ground if all rules in $\cP$ are ground.


%An objective literal $O$ is undefined in $\cI$ if $O \neg \hat{\in}

\subsection{Three-Valued Models for Annotated Atoms} 

Our attention is restricted to three-valued (partial) interpretations
and models such as those extending the well-founded model.  Each such
interpretation $\cI$ is represented as a pair of sets of ground
objective literals: $\cT$ and $\cF$.

%{\sc TES: In the following, I dont yet feel confident about my $>$'s and my
%$\geq$'s.  ... conflicting truth values are mapped to U, but in the
%edge case, I'd like true to outweigh false (as with $r$ below).}

\begin{definition} \label{def:satisfiable}
Let $P$ be a ground objective literal $O$ with
% $atom(O) = A$ and 
$annotation(O) = n$, and $\cI$ an interpretation.

$O$ is true in $\cI$ if
\begin{itemize}
\item $\cT$ contains an $O_T$ such that $O_T$ is a homolog of
  $O$ and $annotation(O_T) \geq n$; and
\item $\cT$ does not contain a conjugate $O_C$ of $O$ with
  $annotation(O_C) \geq (1-n)$; and
\item $\cF$ does not contain a homolog $O_F$ of $O$ with
  $annotation(O_F) \geq (1-n)$.
\end{itemize}

%----------------------------------------------------------------                
$O$ is false in $\cI$ if 
\begin{itemize}
\item $\cT$ does not contain a homolog $O_T$ of $O$ with
  $annotation(O_T) \geq (1-n)$; and either
\begin{itemize}  
\item $\cF$ contains a homolog $O_F$ of $O$ with $annotation(O_F) \geq
  n$; or 
\item $\cT$ contains a conjugate $O_C$ of $O$ with $annotation(O_C) \geq n$.
%Either $A:n_1 \in \cF$ for $n_1 \geq m$ or $\ourneg A:n_2 \in \cT$ for
%$m_2 \geq m$; and
\end{itemize}
\end{itemize}
$O$ is $\ourU$ in $\cI$ if it is neither true nor false in $\cI$.

A positive literal $O$ is true (false) in $\cI$ if $O$ is true (false)
in $\cI$; a negative literal $naf O$ is true in $\cI$ if $O$ is false
in $\cI$ and is false in $\cL$ if $O$ is true in $\cI$.  A literal is
$\ourU$ in $\cI$ if it is neither true nore false in $\cI$.
\end{definition}
%----------------------------------------------------------------              

%\begin{definition} \label{def:satisfiable}
%A ground annotated atom $A:m$ is true in $\cI$ if 
%\begin{itemize}
%\item There exists an $A:n_1 \in \cT$ with $n_1\geq m$, and 
%\begin{itemize}
%\item $\cT$ does not contain $\ourneg A:n_2$ with $n_2 > (1-m)$ and
%\item   $\cF$ does not contain $A:n_3$ with $n_3 > (1-m)$.
%\end{itemize}
%\end{itemize}
%A ground annotated atom $A:m$ is false in $\cI$ if 
%\begin{itemize}
%\item Either $A:n_1 \in \cF$ for $n_1 \geq m$ or $\ourneg A:n_2 \in \cT$
%  for $m_2 \geq m$; and
%\item $\cT$ does not contain $A:n_3$ for $n_3 \geq (1-m)$.
%\end{itemize}
%\end{definition}

Note that in the above definition the truth value \ourU{} captures
both the traditional case where a literal is undefined, along with the
case where a literal is overdefined.

\begin{example} 
  Consider the interpretation $\cI_1$ where
\begin{itemize}
\item  $\cT = \{p:0.7,q:0.5, r:0.6,\ourneg r:0.5\}$
\item $\cF = \{p:0.4, q:0.5\}$.
\end{itemize}
$p:0.6$ is true in $\cI$ and $p:0.8$ false, but $p:0.7$ is \ourU ;
$q:0.5$ is \ourU{} in $\cI$. $r:0.4$ is true in $\cI$ but $r:0.7$ is
false; $r:n$ is \ourU{} for $0.5 \leq n \leq 0.6$.
\end{example}

%\cT$ and $O \neg \hat{\in} \cF$.

%----------------------------------------------------------------

Given Definition~\ref{def:satisfiable}, the definition of the
reduction of a program is the same as for normal logic progams.

\begin{definition}[Reduction of $\cP$ modulo $\cI$] \label{def:reduction}
%                                                                                
Let $\cI$ be an interpretation and $\cP$ a program, both over the same
langage $\cL$.  By the {\em reduction of $P$ modulo $\cI$} we mean a
new program $\frac{P}{\cI}$ obtained from $P$ by performing the
following operations:
\begin{enumerate}
\item Remove from $\cP$ all rules that contain a literal that is false
  in $\cI$.
\item Remove from all the remaining rules those literals that are true
  in $\cI$
\end{enumerate}
\end{definition}

TES: not sure this def is necessary.

\begin{definition}[Unfounded Sets]
Given a program $\cP$, there is a dependency edge from an objective
literal $O_1$ to an objective literal $O_2$ if there is a rule $R \in
\cP$ such that $O_1$ is the head of $R$ and $O_2$ occurs in a literal
$L$ in the body of $R$.  If $O_2$ occurs in a positive default literal
the edge is positive; if $O_2$ occurs in a negative default literal
the edge is negative.

There is a path between objective literals $O_1$ and $O_2$ in $\cP$ if
there is an edge bwtween $O_1$ and $O_2$ in $\cP$, or if there is a
path between $O_1$ and an objective literal $O_3$ and there is an edge
between $O_3$ and $O_2$.

The direct dependencies of an objective literal $\cO$ are those
literals to which $\cO$ has a dependency edge.  
%The dependencies of $\cO$ are those literals to which there is a path
from $\cO$.

An objective literal $O$ is involved in a negative loop if there is a
path from $O$ to itself that involves a negative edge.

An objective literal $O$ is {\em unfounded} if $O$ is involved in a
negative loop and every direct dependency of $O$ is involved in a
negative loop; or if every rule with head $O$ has a non-empty body,
and every direct dependency of $O$ is unfounded.

Given a program $\cP$ and interpretation $\cI$, the {\em unfounded
  set} $\cU_{\cI}$ consists of all unfounded objective literals in
$\frac{\cP}{\cI}$.

\end{definition}

Unfounded sets will be used to separate the truth values of
conjugates.  If the semantics is equivalent to rewriting each rule $r$
for $p$ so that $\naf conjugate(p)$ is in the body of $r$ then a
program like

\begin{verbatim} 
p.
neg p :- not neg p.
\end{verbatim}

\noindent
will be able to derive neither p nor neg p.

%I believe that this is what is done in WFSX, but this seems a bit
%weaker than necessary.  Essentially, we can conclude an objective
%literal if we know that its conjugate is essentially undefined (as
%below).  And we can determine this easily enough if its conjugate has
%been completely evaluated and yet is still undefined.

%----------------------------------------------------------------                 
\subsection{Well-Founded Model}

Motivation: consider the program
\begin{verbatim}
p:0.8:- not p:0.8.
p:0.5.
neg p:0.3.
\end{verbatim}
We want a model with p:0.5 as true, p:0.7 as false and p:0.6 as u.
However, if the rule {\tt p:0.8:- not p:0.8.} were removed, p:0.6
would be false.

\comment{
For two interpretations, $\cI$ and $\cJ$, $\cI
\subseteq \cJ$ iff $true(\cI) \subseteq true(\cJ)$ and $false(\cI)
\subseteq false(\cJ)$.  Alternatively, a three-value interpretation
can be represented as a set of literals.

A program $P$ is {\em safe} if each rule $r$ in $P$ is such that every
variable in $r$ occurs in a positive literal in the body of $r$.  

Symbols within a term may be represented through {\em positions} which
are members of the set $\Pi$.  A {\em position\/} in a term is either
the empty string $\Lambda$ that reaches the root of the term, or the
string $\pi.i$ that reaches the $i$th child of the term reached by
$\pi$, where $\pi$ is a position and $i$ an integer.  For a term $t$
we denote the symbol at position $\pi$ in $t$ by $t_\pi$.  For
example, $p(a,f(X))_{2.1} = X$.  We assume that a program $P$ is
defined over a language $\cL$, containing a finite set $\functions$ of
predicate and function symbols, and a countable set of variables from
the set ${\cal V} \cup \posvar$.  Elements of the set $\cal V$ are
referred to as {\em program variables}.  Elements of the set
$\posvar$, called {\em position variables}, are of the form $X_\pi$,
where $\pi$ is a position.
%
These variables are used when it is convenient to mark certain
positions of interest in a term.  The Herbrand Universe of $\cL$ is
denoted $\cH_{\cL}$, or as $\cH_{P}$ if $\cL$ consists of the
predicate and function symbols in $P$; similarly the Herbrand Base is
denoted as $\cB_{\cL}$ or as $\cB_P$.  Throughout the paper variant
terms are considered to be equal.
}

\subsubsection{Dynamic Stratification} \label{dyn-strat}

One of the most important formulations of stratification is that of
{\em dynamic} stratification.  \cite{Przy89d} shows that a program has
a 2-valued well-founded model iff it is dynamically stratified, so
that it is the weakest notion of stratification that is consistent
with the well-founded semantics.
%
As presented in~\cite{Przy89d}, dynamic stratification computes strata
via operators on interpretations of the form $\langle Tr;Fa \rangle$,
where $Tr$ and $Fa$ are subsets of $\cH_P$.
%

Given a set $\cS$ of ground objective literals, a ground objective
literal $A:m \hat{\in} \cS$ if $A:n \in \cS$ with $n \geq m$.


??? are rules defined properly???

%-----------------------------------------------------------------------
\begin{definition} \label{def:dyn-ops}
For a normal program $P$, sets $Tr$ and $Fa$ of ground atoms and a
3-valued interpretation $I = (\cT,\cF)$ (sometimes called a pre-interpretation):
\begin{description} \item[$True^P_I(Tr) =$]
    $\{A|A$ is not true in $I$;  and 
                        there is a clause
                        $B \leftarrow L_1,...,L_n$
                in $P$, a grounding substitution $\theta$ such that
                $A = B\theta$ and for every $1 \leq i \leq n$ either
                $L_i\theta$ is true in $I$, or $L_i\theta \in Tr$\};
  \item[$False^P_I(Fa) =$] 
%$\{A : val_I(A) \neq {\tt f}$ 
$\{A|A$ is not false in $I$; and for every
    clause $B \leftarrow L_1,...,L_n$ in $P$ and grounding substitution
    $\theta$ such that $A = B\theta$ there is some $i$ $(1 \leq i \leq
    n)$ such that $L_i\theta$ is false in $I$ or $L_i\theta \in Fa\}$.
\end{description}
\end{definition}
%------------------------------------------------------------------------()     
%
\cite{Przy89d} shows that $True^P_I$ and $False^P_I$ are both
monotonic, and defines $\kcaltrue^P_I$ as the least fixed point of $True^P_I(\emptyset)$
and $\calfalse^P_I$ as the greatest fixed point of
$False^P_I(\cH_P)$.
%\footnote{Below, we will sometimes omit the program $P$ in
%  these operators when the context is clear.}.
%, along with an
%operator $\cal I$ that assigns to every interpretation $I$ of $P$ a
%new interpretation ${\cal I}(I) = I \cup \langle \cT_I ; \cF_I \rangle$.
%
In words, the operator $\kcaltrue^P_I$ extends the interpretation $I$ to add
the new atomic facts that can be derived from $P$ knowing $I$; $\calfalse^P_I$
adds the new negations of atomic facts that can be shown false in $P$
by knowing $I$ (via the uncovering of unfounded sets).  An iterated
fixed point operator builds up dynamic strata by constructing
successive partial interpretations as follows.
%----------------------------------------------------------------------          
\begin{definition}[Iterated Fixed Point and Dynamic Strata]
\label{def:IFP}
For a normal program $P$ let 

\begin{center}
$  \begin{array}{rcl}
          WFM_0 & = & \langle \emptyset ; \emptyset \rangle;      \\
 WFM_{\alpha+1} & = &       WFM_{\alpha} \cup
                                \langle \kcaltrue^P_{WFM_\alpha};\calfalse^P_{WFM_\alpha} \rangle; \\
     WFM_\alpha & = & \bigcup_{\beta < \alpha} WFM_\beta, \mbox{ for limit ordinal }\alpha.
  \end{array}
$
\end{center}

\noindent
  $WFM(P)$ denotes the fixed point interpretation $WFM_\delta$,
  where $\delta$ is the smallest (countable) ordinal such that both
  sets $\kcaltrue^P_{WFM_\delta}$ and $\calfalse^P_{WFM_\delta}$ are empty.
%($\delta$ exists, and is
%  a countable ordinal because both $\kcaltrue_I$ and $\calfalse^P_I$ are monotonically
%  increasing).  
% We refer to $\delta$ as the {\em depth} of program $P$.  
The {\em stratum} of atom $A$, is the least ordinal $\beta$ such that
   $A \in WFM_{\beta}$.
% (where $A$ may be either in the true or false
%   component of $WFM_{\beta}$).
\end{definition}
%------------------------------------------------------------------------()      
%
\cite{Przy89d} shows that %the iterated fixed point 
$WFM(P)$ is in fact the well-founded model and that any undefined
atoms of the well-founded model do not belong to any stratum --
i.e. they are not added to $WFM_{\delta}$ for any ordinal
$\delta$. Thus, a program is \emph{dynamically stratified} if every
atom belongs to a stratum.

%------------------------------------------------------------------------
% taking out fixed-order stuff until we put LPADs in (though maybe we
% wont even need it then )

%\input{fixed-order-dynstrat}

